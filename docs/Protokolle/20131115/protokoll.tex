\documentclass[a4paper]{article}

\usepackage[utf8]{inputenc}

\usepackage[parfill]{parskip}

\setlength{\parindent}{0pt}

\begin{document}

\textbf{Nummer:} 6

\textbf{Zeit:} 15.11.13, 12:00-

\textbf{Anwesend:} Lukas, Felix, Daniel, David (Protokollant), Jonas

\textbf{Nächstes Treffen:} 18.11.13, 15:35 Uhr Informatikbau

\section{Inhalt}

Entwerfung der GUI, sowie Verbesserungen am Pflichtenheft. Als UML-Tool wurde Bouml 4.23 gewählt.

\section{Sebastians Verbeserungen übernehmen}

Sebastians Verbesserungen des Pflichtenheft wurden eingepflegt.

\section{Entwurf der GUI}

Lukas und Felix haben einen Entwurf der GUI angefertigt und werden über das Wochenende versuchen einen Protoyp zu erstellen.

\section{Aufgaben bis Mittwoch 13.11.2013}

\begin{itemize}
	\item Produktleistung Zeit testen
	\item Produkteinsatz verbessern (Jonas)
	\item GUI Prototyp fertig machen (Felix)
	\item Qualitätsanforderungen schreiben (Lukas)
	\item Testfälle und Testszenarien schreiben (David)
	\item Grafische Benutzeroberfläche schreiben (Felix)
	\item Systemmodelle schreiben (Jonas)
	\item Glossar schreiben (Daniel)
         \item Logo designen (Daniel)
         
\end{itemize}
\end{document}